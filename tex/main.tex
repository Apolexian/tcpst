\documentclass{article}

\usepackage[backend=bibtex]{biblatex}
\usepackage{color}
\usepackage[nottoc]{tocbibind}
\usepackage{graphicx}
\usepackage{amsthm}
\usepackage{mathtools,textgreek}
\usepackage{amsmath,amsfonts,amsbsy,amssymb}
\usepackage{geometry}
\usepackage{namespc}
\usepackage{cmll}
\usepackage{float}

\addbibresource{ref.bib}

\graphicspath{ {./assets/} }

\newcommand{\todo}[1]{}
\renewcommand{\todo}[1]{{\color{red} TODO: {#1}}}

\newcommand{\sep}{\;\mid\;}

\title{tcpst todo}
\author{todo}
\date{\today}

\begin{document}

\maketitle

\section{intro}

\todo{
    \begin{itemize}
        \item Something on IETF and RFCs
        \item Motivation behind formalising RFCs
        \item Session types (possibly how is this different to other formalisms?)
    \end{itemize}
}

\section{Background}

\todo{
    \begin{itemize}
        \item some info on other formalisms and why STs are interesting, basically motivation for using STs
        \item ST - is there a point in explaining binary?
        \item MPST - is there a point in explaining classic vs less is more?
        \item could include syntax of less is more for context 
        \item should mention that the tool works by translating ST context to mcrl2 checker (not sure if in too much detail)
        \item how are STs useful in PLs? Could include snippet here from my Rust lib as examples
        \item how are STs useful outside of PLs? I.e. that less is more has the model checking tool
    \end{itemize}
}

\section{Discussion/models}

\todo{
    \begin{itemize}
        \item explain what properties we are checking for and what they mean
        \item pop3 model - we can see that its df, all versions of liveness and safe, could be non terminating which is fine
        \item tcp model - harder protocol, calculus cant describe it fully, not df hence lose over properties
        \item why its still useful
        \item why are we not using some calculus with time-outs? (tooling non existent, unclear what these languages mean by timeout)
        \item what other stuff is missing - what can we model using STs and what can we not?
        \item difference in how RFCs are designed and STs - non state based, continuation can make things not actually parallel (especially without async)
        \item what is not done by current ST theory? E.g. The ST concept of time mostly is based on a timeout - how do we describe things like in X time we can send at most Y segments?
        \item how easy is it to solve this discrepancy using some IR? (\todo{if this point is raised, should probably actually make such an IR?})
        \item how do we see people using STs for RFCs - should probably make a case for this being hidden away and generated automatically?
        \item could make some case for some syntax sugar suggestions and modify syntax to include this but this is a dubious contribution?
    \end{itemize}
}

\section{Future directions}

\todo{
    \begin{itemize}
        \item balancing static and dynamic checks
        \item context free?
        \item a more "practical" calculus
        \item parsing and code generation, running model checker on translated code (e.g. in Rust) rather than on sepp ST language
        \item feedback for RFC maker - how can we use this output to direct the document writing? Can we extract the exact point where something went wrong?
        \item e.g. we do something and the model is no longer safe, can we say why?
        \item if we want to include the binary ST library then could talk about how systems languages are suited for STs?
        \item translating ST type contexts to other model checkers?
    \end{itemize}
}

\section{Random bits that will go somewhere}

A new theory of multiparty session types (LM) developed by Scalas and Yoshida~\cite{SY19} removes the concepts of global types and duality.
Instead, LM introduces the concept of behavioural type-level properties.
This change leads to a less restrictive subject reduction that provides a more general theory than classic multiparty session types.

\todo{explain behavioural type-level properties}

\todo{explain how LM is linked to the model checker and introduce the tool}

\todo{explain what benefits model checker has}


\begin{figure}[H]
    \centering
        $
        \begin{array}{rcl}
        \ensuremath{c}
            & \Coloneqq & \ensuremath{s[r] \colon S} \cr
        \ensuremath{S}
            & \Coloneqq & \ensuremath{r \oplus \{l_i(P_i).S_i\}_{i \in I}}
            \sep        \ensuremath{r \with \{l_i(P_i).S_i\}_{i \in I}}
            \sep        \ensuremath{\mu(t).(S)}
            \sep        \ensuremath{t}
            \sep        \ensuremath{end} \cr
        \ensuremath{P}
            & \Coloneqq & \ensuremath{\{TcpPayloadTypes\} \cup \{Pop3PayloadTypes\}}
        \end{array}
        $
    \caption{The syntax of session type contexts used to create the protocol models in \ensuremath{mpstk}.}
    \label{fig:syntax}
\end{figure}

Using the \ensuremath{mpstk} tool we have created models of the POP3 and TCP network protocols.

\todo{explain why TCP is multiparty, check: is this the first ST model that considers TCP to be multiparty?}

The syntax for the session type language that the protocols are written in is presented in Figure~\ref{fig:syntax}.
We do not omit this as the syntax of the language used for \ensuremath{mpstk} differs to that of LM.
Additionally, we only need to consider a simplified subset of the calculus.
Note that \ensuremath{P} can not be an instance of \ensuremath{S}, that is, we do not consider sending sessions across channels.
Although this is standard to the theory of \ensuremath{\pi}-calculus, the presented network protocols would not benefit from such a mechanism.
Hence, the payload types are restricted to the defined TCP and POP3 payload types.

\medskip
\nocite{*}
\printbibliography
\end{document}
