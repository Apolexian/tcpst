\documentclass{article}

\usepackage[backend=bibtex]{biblatex}
\usepackage{color}
\usepackage[nottoc]{tocbibind}
\usepackage{graphicx}
\usepackage{amsthm}
\usepackage{mathtools,textgreek}
\usepackage{amsmath,amsfonts,amsbsy,amssymb}
\usepackage{geometry}
\usepackage{namespc}

\addbibresource{ref.bib}

\graphicspath{ {./assets/} }

\newcommand{\todo}[1]{}
\renewcommand{\todo}[1]{{\color{red} TODO: {#1}}}

\providecommand{\tyselect}[3][]{\ensuremath{{#2}\mathbin{\oplus^{#1}}({#3})}}
\newcommand{\sep}{\;\mid\;}

\title{tcpst todo}
\author{todo}
\date{\today}

\begin{document}

\maketitle

\section{Introduction}

\section{Background}

\subsection{Internet Protocol Standardisation}

\subsection{Session Types}

\todo{binary explanation?}

Multiparty session types (MPST) are a typing discipline that allows us to ensure that a protocol adheres to its session type.
As the name suggests, the theory of MPST focuses on communication between multiple participants.
The classic theory of MPST~\cite{todo} the communication is described from a global view and then projected to the local view of each participant.
\todo{example of classic mpst}.
In this work we use the more general multiparty session type theory (LM) presented by Scalas and Yoshida~\cite{SY19}.
Unlike the classic theory, LM does not base itself on the concepts of global types and duality.
This makes the LM calculus more general - it is able to type more processes.
In addition to this, LM solves the subject reduction flaws that were present in classic MPST theory.

The LM view of multiparty session types is based on the concept of parametrised safety invariants.
This parameter can be adjusted to enforce a number of run-time properties such as deadlock freedom and liveness.

The protocol presented in \todo{the previous example of classic MPST} can be re-written in LM to produce the following typing context.

\todo{same protocol in LM}

If we have a typing context for a protocol that holds with an instantiated safety invariant, then we can say that the run-time process types under this context will hold a similar run-time property.

For example, \todo{explain on example}.

\section{Formalising Internet Protocols Using Session Types}

\todo{
    \begin{itemize}
        \item extracting the typing context from the RFC - discussion on created IR
        \item checking typing context for safety properties
        \item generating code based on typing context that will hold these properties
    \end{itemize}
}

\section{Discussion On Limitations Of The Theory}

\todo{
    \begin{itemize}
        \item We can't express notion of throughput
        \item Generally unclear how we should describe "how much data is being sent" - is this the number of send calls or does the message need to be extended to express this?
        \item There are calculi that do timed and timeouts but these do not have the tooling
        \item Implementation limitations - heterogeneous channels (whats the overhead?), no variadic generics in Rust, will probably encounter other
    \end{itemize}
}

\section{Future Directions}

\todo{
    \begin{itemize}
        \item other properties we may want to parametrise?
        \item model checker reporting - where is property violated?
        \item considering other protocols to exploit other theory of STs? - context free, multiplicities?
    \end{itemize}
}

\[
\begin{array}{rcl}
  \ensuremath{c}
    & \Coloneq & \ensuremath{s[r] \colon S} \cr
  \ensuremath{r}
    & \Coloneq & \ensuremath{v} \cr
  \ensuremath{S}
    & \Coloneqq & \ensuremath{r \oplus \{l_i(P_i).S_i\}_{i \in I}}
    \sep        \ensuremath{r \& \{l_i(P_i).S_i\}_{i \in I}}
    \sep        \ensuremath{\mu(t).(S)}
    \sep        \ensuremath{t}
    \sep        \ensuremath{end} \cr
  \ensuremath{P}
    & \Coloneq & \ensuremath{\{TcpPayloadTypes\} \cup \{Pop3PayloadTypes\}}
\end{array}
\]

We only need to consider a simplified subset of the \pi-calculus.
Specifically, we do not consider that a session itself can be sent on a channel.
This choice is made as the presented network protocols would not benefit from such a mechanism.

\medskip
\nocite{*}
\printbibliography
\end{document}
