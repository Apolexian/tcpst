\begin{xltabular}{\textwidth}{asf}
    \caption{Lorem ipsum}
    \label{table:model_rfc_match} \\
    \hline \\
    Line Number & Model Snippet & RFC Extract\\\hline
    \RaggedRight{$1, 16, 77, 46$}&
    \RaggedRight{{\serversystem \type{<+>tcb\_new(TcbInfo)}} \serveruser \type{\&tcb\_new(TcbInfo)} \clientsystem \type{<+>tcb\_new(TcbInfo)} \clientuser \type{\&tcb\_new(TcbInfo)}}&
    \RaggedRight{"Create a new transmission control block (TCB) to hold connection state information..."}\\\hline
    \RaggedRight{$17, 3$}&
    \RaggedRight{\serveruser \type{<+>error(DiffservSecurity)} \serversystem \type{\&error(DiffservSecurity)}}&
    \RaggedRight{"Verify the security and Diffserv value requested are allowed for this user, if not, return "error: Diffserv value not allowed" or "error: security/compartment not allowed""}\\\hline
    \RaggedRight{$49, 80$}&
    \RaggedRight{\clientuser \type{<+>error(RemoteUnspecified)} \clientsystem \type{\&error(RemoteUnspecified)}}&
    \RaggedRight{"If active and the remote socket is unspecified, return "error: remote socket unspecified""}\\\hline
    \RaggedRight{$47, 78$}&
    \RaggedRight{\clientuser \type{<+>error(ConnectionIllegal)} \clientsystem \type{\&error(ConnectionIllegal)}}&
    \RaggedRight{"If the caller does not have access to the local socket specified, return "error: connection illegal for this process"."}\\\hline
    \RaggedRight{$48, 79$}&
    \RaggedRight{\clientuser \type{<+>error(Insufficient...)} \clientsystem \type{\&error(Insufficient...)}}&
    \RaggedRight{"If there is no room to create a new connection, return "error: insufficient resources""}\\\hline
    \RaggedRight{$50, 19$}&
    \RaggedRight{\serversystem \type{<+>syn(SegSynSet)} \clientsystem \type{\&syn(SegSynSet)}}&
    \RaggedRight{"...if active and the remote socket is specified, issue a SYN segment."}\\\hline
    \RaggedRight{$19, 50$}&
    \RaggedRight{\clientsystem \type{<+>syn\_ack(SegSynAckSet)} \serversystem \type{\&syn\_ack(SegSynAckSet)}}&
    \RaggedRight{"...TCP Peer B sends a SYN and acknowledges the SYN it received from TCP Peer A..."}\\\hline
    \RaggedRight{$53, 21$}&
    \RaggedRight{\serversystem \type{<+>{acceptable(SegAckSet)} \clientsystem \type{\&{acceptable(SegAckSet)}}}}&
    \RaggedRight{"...TCP Peer A responds with an empty segment containing an ACK for TCP Peer B's SYN..."}\\\hline
    \RaggedRight{$19, 50$}&
    \RaggedRight{\type{mu(t)(...)}}&
    \RaggedRight{"Once the connection is established, data is communicated by the exchange of segments..."}\\\hline
    \RaggedRight{$34, 64$}&
    \RaggedRight{\clientsystem \type{<+>rto\_exceeded(SegAckSet)} \serversystem \type{\&rto\_exceeded(SegAckSet)}}&
    \RaggedRight{"For any state if the retransmission timeout expires on a segment in the retransmission queue, send the segment at the front of the retransmission queue again..."}\\\hline
    \RaggedRight{$26, 61$}&
    \RaggedRight{\clientsystem \type{<+>retry\_thresh(SegRstSet)} \serversystem \type{\&retry\_thresh(SegRstSet)}}&
    \RaggedRight{"When the number of transmissions of the same segment reaches a threshold R2 greater than R1, close the connection."}\\\hline
    \RaggedRight{$88, 69$}&
    \RaggedRight{\clientsystem \type{<+>close\_init(Close)} \serversystem \type{<+>fin(SegFinSet)}}&
    \RaggedRight{"The user initiates by telling the TCP implementation to CLOSE the connection..."}\\\hline
    \RaggedRight{$8, 29$}&
    \RaggedRight{\serversystem \type{<+>close\_init(Close)} \clientsystem \type{<+>fin(SegFinSet)}}&
    \RaggedRight{"The remote TCP endpoint initiates by sending a FIN control signal..."}\\\hline
    \RaggedRight{$31, 71$}&
    \RaggedRight{\clientsystem \type{\&fin(SegFinSet)} \serversystem \type{\&fin(SegFinSet)}}&
    \RaggedRight{"Both users CLOSE simultaneously"}\\\hline
    \RaggedRight{$71, 85$}&
    \RaggedRight{\clientuser \type{<+>close(Close).end} \clientsystem \type{\&close(Close)}}&
    \RaggedRight{"When a connection is closed actively, it MUST linger in the TIME-WAIT state..."}\\\hline

\end{xltabular}